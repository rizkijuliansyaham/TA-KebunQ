
\chapter*{ABSTRAK}
\noindent IMPLEMENTASI METODE \textit{RAPID APPLICATION DEVELOPMENT} (RAD) DALAM PERANCANGAN DAN PEMBUATAN APLIKASI MOBILE KEBUNQ BPP LAMPUNG\\
Rizki Juliansyah

\begin{singlespace}
    \begin{justify}
        
    Sektor pertanian merupakan sumber daya alam yang seharusnya dikelola dengan sebaik-baiknya. Urgensi pengoperasian teknologi yang efektif mempengaruhi produktivitas 
    pertanian, yaitu mempermudah pekerjaan petani sehingga memakan waktu yang tidak lama serta tidak dibutuhkannya lagi tenaga kerja manual.
    Balai Pelatihan Pertanian (BPP) Lampung juga masih mengharuskan tenaga kerja datang ke lokasi untuk melakukan \textit{monitoring} 
    dan kontrol kondisi lahan. Ketersediaan \emph{smartphone} di kalangan petani dapat memberikan dampak positif yaitu peningkatan produktivitas pertanian melalui
    penerapan Teknologi Informasi dan Komunikasi. Berdasarkan hal tersebut dilakukan perancangan dan pembuatan aplikasi KEBUNQ. \emph{System Development Life Cycle} (SDLC) yang digunakan adalah \emph{Rapid Application Development}
    (RAD). Pengujian yang dilakukan menggunakan metode \emph{black box testing} untuk menguji fungsionalitas aplikasi, dan \emph{User Acceptance Testing} (UAT) untuk mengetahui tingkat penerimaan pengguna. Pada penelitian ini, \emph{black box} didapatkan satu bagian yang belum sesuai dengan yang diharapkan. Kemudian pada pengujian UAT, didapatkan hasil 89,5\% yang menunjukkan tingkat penerimaan yang sangat kuat oleh pengguna.\\[2cm]
        \textbf{Kata Kunci: Pertanian, \textit{Smartphone, Rapid Application Development, Black box testing, User Acceptance Testing}}
    \end{justify}

\end{singlespace}

\newpage