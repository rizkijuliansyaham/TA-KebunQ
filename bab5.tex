%-----------------------------------------------------------------------------%
\chapter{KESIMPULAN DAN SARAN}
%-----------------------------------------------------------------------------%

%
\vspace{4.5pt}

\begin{flushleft}
    \begin{justify}
        \section{Kesimpulan}
        Berdasarkan hasil penelitian yang dilakukan, maka dapat disimpulkan sebagai berikut:
        \begin{enumerate}
            \item Aplikasi KEBUNQ berhasil dibuat dan berjalan di android.
            \item Aplikasi KEBUNQ yang dibuat dapat melakukan \emph{monitoring} dan kontrol pada alat yang dipasang.
            \item Pengimplementasian RAD mempermudah peneliti dalam melakukan manajemen proyek.
            \item Hasil pengujian yang tidak memenuhi harapan dan dikarenakan faktor eksternal (berada diluar \emph{jobdesk} peneliti sebagai mobile \emph{developer}) 
            menjadi bahan pengajuan perbaikan kepada BPP Lampung. 
            \item Hasil pengujian UAT aplikasi KEBUNQ memiliki nilai persentase 92,2\% dengan keterangan penerimaan sangat kuat, hal ini merepresentasikan bahwa aplikasi dapat diterima oleh responden atau pengguna.\\
        \end{enumerate}

        \section{Saran}
        Berdasarkan aplikasi KEBUNQ yang dirancang dan dibuat masih sederhana perlu adanya pengembangan yang dilakukan untuk penelitian selanjutnya, sehingga peneliti memberikan saran sebagai berikut:
        \begin{enumerate}
            \item Perlu adanya peningkatan pengamanan pada aplikasi.
            \item Perlu adanya pengembangan \emph{User Interface} (UI) yang lebih baik dengan kaidah dan aturan yang ada.
            \item Perlu adanya pengembangan fitur dengan mempertimbangkan aspek \emph{User Experience} (UX).
            \item Dalam penerapan RAD, benar-benar dibutuhkan tim yang cepat dan solid, sehingga ketika mengalami masalah dapat diperbaiki dengan cepat.
        \end{enumerate}

            
    \end{justify}
        
\end{flushleft}

\newpage