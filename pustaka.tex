% Daftar Pustaka IEEE
% BUKU
%  [1] Penulis, 
% Judul Buku. 
% Edisi.
% Kota Terbit : 
% Nama Penerbit, 
% Tahun Terbit.
% 
% 
% CONTOH :

% [1]  Oliviero, Andrew and Woodward, Bill, “Cable Design” in Cabling The Complete Guide To Copper and Fiber
% Optic Networking, 4th ed. United State of America : Wiley Publishing, Inc, 2009. pp. 19 – 33.
% [2]  J. Moran, Michael and Shapiro, H.N., Fundamentals Of Engineering Thermodynamics, 2nd ed. United States of America : John Wiiley and Son, 1993.
% [3]  B. Stanley, et al., C++ Primer, Fourth Edition, Massachusetts : Addison Wesley Professional, 2005.
% 
% 
% 
% 
% 
% 
% 
% 
% 
% Daftar Pustaka 
% 
%Format Penulisan Daftar Pustaka yang di Sesuai dengan aturan penulisan Unair
%==========================================================================================
%\bibitem[Nama,tahun]{citasi}
%Nama Pengarang, Tahun,
%\newblock \emph{Judul dan keterangan}.
%\newblock Penerbit, Kota.
%========================================================================================== 
% Tambahkan pustaka yang digunakan setelah perintah berikut. 
% 


\begin{thebibliography}{7}
%\bibliographystyle{unsrt}
\providecommand{\natexlab}[1]{#1}
\providecommand{\url}[1]{\texttt{#1}}
\expandafter\ifx\csname urlstyle\endcsname\relax
  \providecommand{\doi}[1]{doi: #1}\else
  \providecommand{\doi}{doi: \begingroup \urlstyle{rm}\Url}\fi


\bibitem[1]{dokumenBalitbang}
BPTP Balitbangtan Jambi,
\newblock "ALAT DAN MESIN PERTANIAN TEPAT GUNA
UNTUK TANAMAN PADI DALAM MENDUKUNG PROGRAM
PENINGKATAN PRODUKSI BERAS NASIONAL (P2BN),"
\newblock 2020, Available: http://jambi.litbang.pertanian.go.id/ind/images/PDF/Kiki1.pdf

\bibitem[2]{web-datasmartphone}
Yosepha Pusparisa, (2021, July.1)
\newblock \emph{Daftar Negara Pengguna Smartphone Terbanyak, Indonesia Urutan Berapa?}
\newblock [Online]. Available : https://databoks.katadata.co.id/datapublish/2021/07/01/daftar-negara-pengguna-smartphone-terbanyak-indonesia-urutan-berapa

\bibitem[3]{jurnal-kajianAplikasi}
Direktorat Pangan dan Pertanian, “Studi Pendahuluan Rencana Pembangunan Jangka Menengah Nasional (RPJMN) Bidang Pangan dan Pertanian 2015 – 2016”, Direktorat Pangan dan Pertanian Kementerian Perancanaan Pembangunan Nasional/Badan Perencanaan Pembangunan Nasional, 2013.

\bibitem[4]{Sukamto}
Sukamto, R. A. dan Shalahudin, M.. Rekayasa Perangkat Lunak Terstruktur dan Berorientasi Objek. Bandung : Informatika Bandung, 2016.

\bibitem[5]{jurnal empiris}
P. Beynon-Davies, C. Carne, H. Mackay, and D. Tudhope, “Rapid application development (RAD): An empirical review,” \emph{European Journal of Information Systems}, vol. 8, no. 3, pp. 212, 1999. 

\bibitem[6]{API}
IBM Cloud Education, “What is an application programming interface (API),” \emph{IBM}, 19-Aug-2020. [Online]. Available: https://www.ibm.com/cloud/learn/api. [Accessed: 11-Apr-2022]. 

\bibitem[7]{Database}
OCI, “What is a database?,” \emph{Oracle}, 2022. [Online]. Available: https://www.oracle.com/database/what-is-database/. [Accessed: 11-Apr-2022]. 

\bibitem[8]{Flowchart}
“What is a flowchart?,” \emph{ASQ}. [Online]. Available: https://asq.org/quality-resources/flowchart. [Accessed: 12-Apr-2022]. 

\bibitem[9]{mobile}
N. Serrano, J. Hernantes and G. Gallardo. Mobile Web Apps. \emph{IEEE Software}, vol. 30, no. 5, 2013, pp. 22 -27.

\bibitem[10]{flutter}
Ganda, Yusmi P. W., Happy Flutter. Cetakan 1. Tangerang Selatan : Al Qolam, 2019.

\end{thebibliography}

