% Daftar Pustaka IEEE
% BUKU
%  [1] Penulis, 
% Judul Buku. 
% Edisi.
% Kota Terbit : 
% Nama Penerbit, 
% Tahun Terbit.
% 
% 
% CONTOH :

% [1]  Oliviero, Andrew and Woodward, Bill, “Cable Design” in Cabling The Complete Guide To Copper and Fiber
% Optic Networking, 4th ed. United State of America : Wiley Publishing, Inc, 2009. pp. 19 – 33.
% [2]  J. Moran, Michael and Shapiro, H.N., Fundamentals Of Engineering Thermodynamics, 2nd ed. United States of America : John Wiiley and Son, 1993.
% [3]  B. Stanley, et al., C++ Primer, Fourth Edition, Massachusetts : Addison Wesley Professional, 2005.
% 
% 
% 
% 
% 
% 
% 
% 
% 
% Daftar Pustaka 
% 
%Format Penulisan Daftar Pustaka yang di Sesuai dengan aturan penulisan Unair
%==========================================================================================
%\bibitem[Nama,tahun]{citasi}
%Nama Pengarang, Tahun,
%\newblock \emph{Judul dan keterangan}.
%\newblock Penerbit, Kota.
%========================================================================================== 
% Tambahkan pustaka yang digunakan setelah perintah berikut. 
% 


\begin{thebibliography}{7}
%\bibliographystyle{unsrt}
\providecommand{\natexlab}[1]{#1}
\providecommand{\url}[1]{\texttt{#1}}
\expandafter\ifx\csname urlstyle\endcsname\relax
  \providecommand{\doi}[1]{doi: #1}\else
  \providecommand{\doi}{doi: \begingroup \urlstyle{rm}\Url}\fi


\bibitem[1]{dokumenBalitbang}
BPTP Balitbangtan Jambi, "Alat dan Mesin Pertanian Tepat Guna untuk Tanaman Padidalam Mendukung Program Peningkatan Produksi Beras Nasional (P2BN)"
\newblock 2020, Available: http://jambi.litbang.pertanian.go.id/ind/images/PDF/Kiki1.pdf

\bibitem[2]{web-datasmartphone}
Y. Pusparisa, “Daftar Negara Pengguna smartphone Terbanyak, Indonesia urutan berapa?: Databoks,” \emph{Databoks Pusat Data Ekonomi dan Bisnis Indonesia}, 01-Jul-2021. [Online]. Available: https://databoks.katadata.co.id/datapublish/2021/07/01/daftar-negara-pengguna-smartphone-terbanyak-indonesia-urutan-berapa. [Accessed: 06-Feb-2022]. 

\bibitem[3]{jurnal-kajianAplikasi}
Direktorat Pangan dan Pertanian, “Studi Pendahuluan Rencana Pembangunan Jangka Menengah Nasional (RPJMN) Bidang Pangan dan Pertanian 2015 – 2016”, Direktorat Pangan dan Pertanian Kementerian Perancanaan Pembangunan Nasional/Badan Perencanaan Pembangunan Nasional, 2013.
% https://journal.uii.ac.id/Snati/article/download/6236/5598

\bibitem[4]{Sukamto}
Sukamto, R. A. dan Shalahudin, M.. Rekayasa Perangkat Lunak Terstruktur dan Berorientasi Objek. Bandung : Informatika Bandung, 2016.

\bibitem[5]{jurnal empiris}
P. Beynon-Davies, C. Carne, H. Mackay, and D. Tudhope, “Rapid application development (RAD): An empirical review,” \emph{European Journal of Information Systems}, vol. 8, no. 3, pp. 212, 1999. 
% https://www.researchgate.net/publication/31978101_Rapid_application_development_RAD_An_empirical_review

\bibitem[6]{web waterfall}
M. P. Putri and H. Effendi, “Implementasi metode RAD Pada website  service guide ‘Tour waterfall south sumatera’,” \emph{Jurnal Sisfokom (Sistem Informasi dan Komputer)}, vol. 7, no. 2, pp. 130–136, Sep. 2018. 
% https://www.google.com/url?sa=t&rct=j&q=&esrc=s&source=web&cd=&cad=rja&uact=8&ved=2ahUKEwjioNOUoeP3AhW88HMBHS9qCqkQFnoECAYQAQ&url=http%3A%2F%2Fjurnal.atmaluhur.ac.id%2Findex.php%2Fsisfokom%2Farticle%2Fview%2F00021&usg=AOvVaw2jJicE3WeLE5gyXS2aFEdB

\bibitem[7]{jurnal RAD UAT}
A. Rahman, "Rapid Application Development Sistem Pembelajaran Daring Berbasis Android," \emph{Jurnal Intech}, vol. 1, no. 2, pp. 20-25, Nov. 2020.
% https://www.google.com/url?sa=t&rct=j&q=&esrc=s&source=web&cd=&cad=rja&uact=8&ved=2ahUKEwitj6T-oOP3AhVS7XMBHf5rDaYQFnoECAYQAQ&url=https%3A%2F%2Fjournal.unbara.ac.id%2Findex.php%2FINTECH%2Farticle%2Fdownload%2F639%2F464%2F&usg=AOvVaw3MNPdSK6Bm-OYVrWqfrTL7

\bibitem[8]{jurnal RAD UAT 2}
D. Aryani, Malabay, H. D. Ariessanti, "Penerapan Rapid Application Development (RAD) Pada Perancangan Aplikasi Tracer Study Berbasis Android," \emph{eDikInformatika}, vol. 7, no. 1, pp. 111-122, Oct. 2020.
% https://digilib.esaunggul.ac.id/public/UEU-Journal-20259-11_1399.pdf

\bibitem[9]{Monitoring}
Hikmat, Dr. Harry. 2010. \emph{Monitoring dan Evaluasi Proyek}.

\bibitem[10]{Kontrol}
Katsuhiko Ogata. "Teknik kontrol automatik“, PT penerbit erlangga-SIMON. \&SCHUTER (ASUA) Pte.ltd., 1997.

\bibitem[11]{mobile}
N. Serrano, J. Hernantes and G. Gallardo. Mobile Web Apps. \emph{IEEE Software}, vol. 30, no. 5, 2013, pp. 22 -27.
% https://www.researchgate.net/publication/313868513_Studying_Mobile_Apps_for_Agriculture

\bibitem[12]{flutter}
Ganda, Yusmi P. W., Happy Flutter. Cetakan 1. Tangerang Selatan : Al Qolam, 2019.

\bibitem[13]{API}
IBM Cloud Education, “What is an application programming interface (API),” \emph{IBM}, 19-Aug-2020. [Online]. Available: https://www.ibm.com/cloud/learn/api. [Accessed: 11-Feb-2022]. 

\bibitem[14]{Database}
OCI, “What is a database?,” \emph{Oracle}, 2022. [Online]. Available: https://www.oracle.com/database/what-is-database/. [Accessed: 11-Feb-2022]. 

\bibitem[15]{Flowchart}
“What is a flowchart?,” \emph{ASQ}. [Online]. Available: https://asq.org/quality-resources/flowchart. [Accessed: 15-Feb-2022]. 

\bibitem[16]{gambar fc}
D. Rizky, “Jenis Flowchart Dan Simbol-Simbolnya,” \emph{Medium}, 30-Apr-2019. [Online]. Available: https://medium.com/dot-intern/jenis-flowchart-dan-simbol-simbolnya-ef6553c53d73. [Accessed: 20-Feb-2022]. 

\bibitem[17]{use case 1}
M. R. Adani, “Use case diagram: Pengertian, Fungsi, Teknik, Dan Contoh,” \emph{Sekawan Media | Software House \&amp; System Integrator Indonesia}, 21-Jun-2021. [Online]. Available: https://www.sekawanmedia.co.id/blog/use-case-diagram/. [Accessed: 20-Feb-2022]. 

\bibitem[18]{use case 2}
Lucid Software, “UML use case diagram tutorial,” \emph{Lucidchart}. [Online]. Available: https://www.lucidchart.com/pages/uml-use-case-diagram. [Accessed: 24-Feb-2022]. 

\bibitem[19]{figma uc}
F. Mandrelli, “UML use case diagram,” \emph{Figma}, 2021. [Online]. Available: https://www.figma.com/community/file/986330591099819762. [Accessed: 24-Feb-2022]. 

\bibitem[20]{buku scholar}
R.   Habibi   and   R.   Aprilian,   Tutorial   dan Penjelasan  Aplikasi  E-Office  Berbasis  Web Menggunakan Metode RAD, Bandung: Kreatif Industri Nusantara, 2020. 
% bukunya https://books.google.co.id/books?hl=id&lr=&id=h5PuDwAAQBAJ&oi=fnd&pg=PR1&dq=related:MMIDeyjXtEUJ:scholar.google.com/&ots=Ht-Xb71R5S&sig=X1LON_cyff00KLfgqs0woyCpU-Q&redir_esc=y#v=onepage&q&f=false
% akun beliau https://scholar.google.co.id/citations?view_op=view_citation&hl=id&user=IBtTXDAAAAAJ&citation_for_view=IBtTXDAAAAAJ:LkGwnXOMwfcC
%  halaman 101

\bibitem[21]{black box}
Iskandaria. 2012. Contoh Pengujian Black Box

\bibitem[22]{uat}
Mutiara, A. B., Awaludin, R., Muslim, A. and T. Oswari, “Testing Implementasi Website Rekam Medis Elektronik Opeltgunasys Dengan Metode Acceptance Testing,” 2014.

\bibitem[23]{likert}
Sugiyono, \emph{Memahami Penelitian Kualitatif}. Bandung, Indonesia: Alfabeta, 2012.

\bibitem[24]{kuantitatif}
Riduwan, \emph{Belajar mudah penelitian untuk guru-karyawan dan peneliti pemula}. Bandung, Indonesia: Alfabeta, 2009. 


\end{thebibliography}

