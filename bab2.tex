%-----------------------------------------------------------------------------%
\chapter{LANDASAN TEORI}
%-----------------------------------------------------------------------------%

%
\vspace{4.5pt}

\begin{flushleft}
    \section{Tinjauan Pustaka}
    \section{Dasar Teori}
    \begin{justify}
        \subsection{Sistem \textit{Monitoring}}
\blindtext

\subsection{Sistem Kontrol}


        \subsection{Aplikasi Mobile}

        \subsection{\textit{Rapid Application Development} (RAD)}

        \subsection{Flutter}

        \subsection{API}

        % \subsection{VPS}

        \subsection{Database}

        \subsection{\textit{Use Case} Diagram}

        \subsection{\textit{Black Box Testing}}

        \subsection{\textit{User Acceptance Testing} (UAT)}
        \noindent User Acceptance Testing (UAT) adalah pengujian terakhir yang dilakukan oleh user secara langsung, dan pada saat pengujian berlangsung pembuatan dokumen juga dilakukan sebagai bukti penerimaan sistem oleh pengguna. 
        \textcolor{red}{[Mutiara, A. B., Awaludin, R., Muslim, A. and T. Oswari, “Testing Implementasi Website Rekam Medis Elektronik Opeltgunasys Dengan Metode Acceptance Testing,” 2014.].
        }
        

        \subsection{Skala \textit{Likert}}

    \end{justify}



\end{flushleft}



\newpage