
% Halaman Abstract

\chapter*{ABSTRACT}
\noindent IMPLEMENTATION OF THE RAPID APPLICATION DEVELOPMENT (RAD) METHOD IN THE DESIGN AND DEVELOPMENT OF MOBILE APPLICATIONS KEBUNQ BPP LAMPUNG\\
Rizki Juliansyah

\begin{singlespace}
% \blindtext \\[20pt]

\begin{justify}
    The agricultural sector is a natural resource that should be managed as well as 
    possible. The urgency of operating an effective technology affects agricultural 
    productivity; namely, making it easier for farmers to work so that it doesn't take 
    long and there is no need for manual labor. The Lampung Agricultural Training Center 
    (BPP) also still requires workers to come to the location to monitor and control land 
    conditions. The availability of smartphones among farmers can have a positive impact, 
    namely increasing agricultural productivity through the application of Information and 
    Communication Technology. Based on this, the design and manufacture of the KEBUNQ 
    application was carried out. The design and manufacture of a software or application 
    cannot be separated from the implementation of the Software Development Life Cycle 
    (SDLC). SDLC aims to maximize the output of the software created. In this study, the 
    SDLC used is Rapid Application Development (RAD). Tests carried out using the black 
    box testing method to test the application's functionality, and User Acceptance 
    Testing (UAT) to determine the level of user acceptance. In this study, black box 
    testing obtained results as expected. Then in the UAT test, the results were 85.77\% 
    which showed a very strong level of acceptance by the user.\\[2cm]
    \textbf{Keywords: Software Development Life Cycle, Smartphone, Rapid Application Development, Black box testing, User Acceptance Testing.}

\end{justify}

    
\end{singlespace}

\newpage