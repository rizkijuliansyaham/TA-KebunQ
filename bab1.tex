%-----------------------------------------------------------------------------%
\chapter{PENDAHULUAN}
%-----------------------------------------------------------------------------%

\vspace{4.5pt}

\begin{flushleft}
  \begin{justify}
    \section{Latar Belakang} 
    Sektor pertanian merupakan sumber daya alam yang seharusnya dikelola dengan sebaik-baiknya. Pengelolaan sektor pertanian yang baik dipengaruhi oleh penggunaan teknologi yang tepat guna dan keefektifan dalam pengoperasiannya. Dalam referensi [] teknologi tepat guna sederhana adalah teknologi yang dibuat atas dasar ketersediaan komponen lokal, dan dapat dikembangkan oleh sumber daya manusia lokal. Namun, pengoperasian teknologi pada sektor pertanian beberapa diantaranya masih memakan waktu yang lama dan menggunakan tenaga kerja manual. Urgensi pengoperasian teknologi yang efektif mempengaruhi produktivitas pertanian, yaitu mempermudah pekerjaan petani sehingga memakan waktu yang tidak lama serta tidak dibutuhkannya lagi tenaga kerja manual. 

    Berdasarkan hasil wawancara terhadap penanggungjawab program smart farming BPP Lampung, menyatakan bahwa pengolahan lahan di Balai Pelatihan Pertanian (BPP) Lampung mengharuskan tenaga kerja datang ke lokasi untuk melakukan \textit{monitoring} kondisi lahan, diantaranya: pengecekan suhu udara, kelembapan udara, intensitas cahaya, suhu air, suhu tanah, ppm air, pH tanah, pH air, kelembapan tanah, dan tekanan udara menggunakan alat pengukur. Selain \textit{monitoring}, dilakukan juga kontrol sistem penyiraman pada lahan. Sistem \textit{monitoring} dan \textit{controlling} tersebut tergolong tidak efektif dikarenakan masih beroperasi menggunakan tenaga kerja manual sehingga memakan waktu yang lama. Maka daripada itu diperlukannya  inovasi yang dapat mendukung keefektifan para petani dalam mengoperasikan teknologi. 

    Teknologi yang berkembang pesat saat ini adalah penggunaan \textit{smartphone}. Dalam referensi [link notes] jumlah pengguna smartphone di Indonesia mencapai 170,4 juta. 19,6\% dari pengguna \textit{smartphone} merupakan petani di Indonesia. Jumlah petani di Indonesia akan terus bertambah mengingat perekonomian nasional sangat bergantung pada sektor pertanian sesuai dengan referensi [jurnal seminar nasional] yang menyatakan bahwa sektor pertanian menyumbang 14,9\% dari Produk Domestik Bruto (PDB). Berdasarkan data tersebut, ketersediaan \textit{smartphone} di kalangan petani Indonesia dapat memberikan dampak positif yaitu peningkatan produktivitas pertanian melalui penerapan Teknologi Informasi dan  Komunikasi (TIK). 

    Berdasarkan permasalahan di atas maka dilakukan penelitian pengembangan dengan judul \textbf{“Implementasi metode \textit{Rapid Application Development} (RAD) dalam perancangan dan pembuatan aplikasi \textit{mobile} KebunQ BPP Lampung”}.  Penelitian ini dilakukan sekaligus untuk membantu program \textit{Low Cost Smart Farming} BPP Lampung. Pemilihan metode RAD pada penelitian ini didasarkan atas ketersediaan waktu pengerjaan yang pendek [Mcld, sukamto] dan jumlah tim yang terbatas [].


  \end{justify}
    

\begin{justify}
  
  
%   Sektor pertanian merupakan sumber daya alam yang seharusnya dikelola dengan sebaik-baiknya. Pengelolaan sektor pertanian yang baik dipengaruhi oleh penggunaan teknologi yang tepat guna dan keefektifan dalam pengoperasiannya. Dalam referensi [] teknologi tepat guna sederhana adalah teknologi yang dibuat atas dasar ketersediaan komponen lokal, dan dapat dikembangkan oleh sumber daya manusia lokal. Namun, pengoperasian teknologi pada sektor pertanian beberapa diantaranya masih memakan waktu yang lama dan menggunakan tenaga kerja manual. Urgensi pengoperasian teknologi yang efektif mempengaruhi produktivitas pertanian, yaitu mempermudah pekerjaan petani sehingga memakan waktu yang tidak lama serta tidak dibutuhkannya lagi tenaga kerja manual. 
% \\
% \\
% Pengolahan lahan di Balai Pelatihan Pertanian (BPP) Lampung mengharuskan tenaga kerja datang ke lokasi untuk melakukan monitoring kondisi lahan, diantaranya: pengecekan suhu udara, kelembapan udara, intensitas cahaya, suhu air, suhu tanah, ppm air, pH tanah, pH air, kelembapan tanah, dan tekanan udara menggunakan alat pengukur. Selain monitoring, dilakukan juga pengkontrolan lahan diantaranya: menghidupkan mesin, membuka keran, mengarahkan selang penyiraman, (dari : Pak Adi Destrianda - Penanggungjawab smartfarming)


%   KEBUNQ buat apa, urgensi, data pendukung\\
%   data lampung, luas, butuh efisiensi, bagaimana pertanian di lampung.\\
%   rumusan masalah dapat mencaapai tujuan dan untuk mencapai tujuan menggunakan metoda yang diteliti\\
%   dengan metode diharapkan ..... manfaat penelitian.
%   \\
%   \\
%   Teknologi memiliki peran besar dalam perkembangan
%   \\
%   \\
%   Balai Penyuluhan Pertanian (BPP) Lampung, khususnya BAGIAN DIVISI SMARTFARMING membutuhkan suatu aplikasi untuk memantau dan melakukan kontrol
%   perangkat IoT yang dibuat dan digunakan pada lahan kebun. Sistem pemantauan dan kontrol yang dilakukan saat ini
%   masih secara konvensional dimana pengurus harus datang ke lahan untuk melihat keadaan lahan dan melakukan penyiraman maupun kontrol lainnya. Kelemahannya adalah tidak flexiblenya pemeliharaan lahan, data keadaan lahan belum terekam,
%   pimpinan tidak dapat memonitor keadaan lahan terkini, dan instansi tidak bisa melakukan evaluasi lahan dan penelitian berdasarkan data rekaman lahan tersebut.
%   \\
%   \\
%   Berdasarkan masalah yang ada maka perlu dirancang dan dibuat aplikasi untuk membantu
% \textit{monitoring} data \textit{realtime}, melakukan kontrol berdasarkan sensor dan aktuator yang tersedia pada alat serta penyediaan database yang dapat menyimpan data rekaman nilai sensor untuk kebutuhan
% penelitian BPP Lampung maupun pihak yang bekerjasama dengan BPP Lampung. \textit{Software Development Life Cycle} (SDLC) yang diimplementasikan adalah
% \textit{Rapid Application Development} (RAD). Penggunaan RAD dalam perancangan dan pembuatan aplikasi KEBUNQ ini merupakan gagasan yang tepat karena sangat cocok untuk 
% waktu pengerjaan yang pendek [McLeod, Management Information Systems. 2001] dan resources team yang terbatas.

\end{justify}
\vspace{1cm}   

\section{Rumusan Masalah}
\begin{justify}
  Berdasarkan latar belakang yang telah dipaparkan, maka rumusan masalahnya adalah sebagai berikut.
\end{justify}
\begin{enumerate}
  \item Bagaimana penerapan metode RAD dalam perancangan dan pembuatan aplikasi mobile KEBUNQ?
  \item Bagaimana membuat aplikasi yang dapat berjalan di andoid?
  \item Bagaimana integrasi antara aplikasi KEBUNQ dengan alat yang ada di lahan?
  
  
\end{enumerate}
\vspace{2cm}

\section{Tujuan Penelitian}

\begin{justify}
  Adapun tujuan dalam penelitian ini adalah sebagai berikut.


\end{justify}
\begin{enumerate}
  \item Menganalisa penerapan RAD dalam pembuatan aplikasi KEBUNQ.
  \item Merancang dan membuat aplikasi KEBUNQ.
  \item Menganalisa integrasi antara aplikasi, alat dan database.
  
  \end{enumerate}
\vspace{2cm}

\section{Batasan Masalah}

\begin{justify}
  Adapun batasan masalah dalam penelitian ini adalah sebagai berikut.


\end{justify}
\begin{enumerate}
  \item Pembahasan hanya berfokus pada perancangan dan pembuatan aplikasi mobile KEBUNQ.
  \item Pengerjaan backend dan alat dikerjakan oleh individu yang berbeda dalam team ini.
  \item Bahasa pemrograman yang digunakan adalah bahasa dart dengan framework flutter.
  \item Release aplikasi yang dibuat hanya terbatas release pada playstore belum pada appstore walaupun pembuatan menggunakan menggunakan bahasa pemrograman \textit{crossplatform}.
  \item Berhubung aplikasi KEBUNQ merupakan project instansi terkait, maka beberapa data tidak dicantumkan pada penelitian ini.

  \end{enumerate}
\vspace{2cm}


\section{Manfaat Penelitian}

\begin{justify}
  Adapun manfaat dari penelitian ini adalah sebagai berikut.


\end{justify}
\begin{enumerate}
  \item Membantu instansi terkait dalam mengelola lahan pertanian.
  \item Dalam pengelolaan lahan, data seperti rekaman nilai sensor diperlukan untuk monitoring kualitas lahan. Dengan adanya aplikasi KEBUNQ, instansi terkait dapat memonitori lahan secara realtime dan data rekamannya dapat digunakan pada penelitian yang dilaksanakan instansi terkait.

  \end{enumerate}
\vspace{2cm}



\section{Sistematika Penulisan}
Pada penelitian ini peneliti menyusun berdasarkan sistematika penulisan sebagai berikut: \\
\begin{itemize}
  \item \noindent BAB I PENDAHULUAN
  \begin{justify}
  Bab ini berisi tentang latar belakang, rumusan masalah, tujuan penelitian, 
  batasan masalah, manfaat penelitian, dan sistematika penulisan.
  \end{justify}
\end{itemize}

\begin{itemize}
  \item \noindent BAB II TINJAUAN PUSTAKA
  \begin{justify}
    Bab ini berisi tentang teori-teori yang mendukung atau berhubungan dengan aplikasi ini.

  \end{justify}
\end{itemize}

\begin{itemize}
  \item \noindent BAB III METODE PENELITIAN
  \begin{justify}
    Bab ini berisi tentang teori-teori yang mendukung atau berhubungan dengan aplikasi ini.

  \end{justify}
\end{itemize}


\begin{itemize}
  \item \noindent BAB IV HASIL IMPLEMENTASI DAN PEMBAHASAN
  \begin{justify}
    Bab ini berisi tentang hasil implementasi dari rancangan penelitian beserta pembahasannya.

  \end{justify}
\end{itemize}


\begin{itemize}
  \item \noindent BAB V KESIMPULAN DAN SARAN
  \begin{justify}
    Bab ini berisi kesimpulan dan saran yang didapatkan dari penelitian ini.

  \end{justify}
\end{itemize}



\end{flushleft}



\newpage