%-----------------------------------------------------------------------------%
\chapter{PENDAHULUAN}
%-----------------------------------------------------------------------------%

\vspace{4.5pt}

\begin{flushleft}
  \begin{justify}
    \section{Latar Belakang} 
    Sektor pertanian merupakan sumber daya alam yang seharusnya dikelola dengan sebaik-baiknya. 
    Pengelolaan sektor pertanian yang baik dipengaruhi oleh penggunaan teknologi yang tepat guna dan keefektifan dalam pengoperasiannya. 
    Penggunaan teknologi yang tepat guna dipengaruhi oleh beberapa aspek lokal, diantaranya adalah aspek lingkungan, aspek sosial (sumber daya manusia lokal), dan aspek ekonomi masyarakat \cite{dokumenBalitbang,teknologi}. Namun, pengoperasian teknologi pada sektor pertanian beberapa diantaranya masih 
    memakan waktu yang lama dan menggunakan tenaga kerja manual. Urgensi pengoperasian teknologi yang efektif mempengaruhi produktivitas 
    pertanian, yaitu mempermudah pekerjaan petani sehingga memakan waktu yang tidak lama serta tidak dibutuhkannya lagi tenaga kerja manual. 

    Berdasarkan hasil wawancara terhadap penanggungjawab program smart farming BPP Lampung, menyatakan bahwa pengolahan lahan di 
    Balai Pelatihan Pertanian (BPP) Lampung mengharuskan tenaga kerja datang ke lokasi untuk melakukan \textit{monitoring} 
    kondisi lahan, diantaranya: pengecekan suhu udara, kelembapan udara, intensitas cahaya, suhu air, suhu tanah, ppm air, 
    pH tanah, pH air, kelembapan tanah, dan tekanan udara menggunakan alat pengukur. Selain \textit{monitoring}, dilakukan juga 
    kontrol sistem penyiraman pada lahan. Sistem \textit{monitoring} dan \textit{controlling} tersebut tergolong tidak efektif 
    dikarenakan masih beroperasi menggunakan tenaga kerja manual sehingga memakan waktu yang lama. Maka daripada itu diperlukannya  
    inovasi yang dapat mendukung keefektifan para petani dalam mengoperasikan teknologi. 

    Teknologi yang berkembang pesat saat ini adalah penggunaan \textit{smartphone}. Dalam referensi \cite{web-datasmartphone} 
    jumlah pengguna smartphone di Indonesia mencapai 170,4 juta. Menurut catatan Badan Pusat Statistik (BPS), jumlah petani per 2019 mencapai 33,4 juta orang \cite{databps}. 
    Jumlah petani di Indonesia akan terus bertambah mengingat perekonomian nasional sangat bergantung pada sektor pertanian sesuai dengan 
    referensi \cite{jurnal-kajianAplikasi} yang menyatakan bahwa sektor pertanian menyumbang 14,9\% dari Produk Domestik Bruto (PDB). Berdasarkan data tersebut, ketersediaan \textit{smartphone} di kalangan petani Indonesia dapat memberikan dampak positif yaitu peningkatan produktivitas pertanian melalui penerapan Teknologi Informasi dan  Komunikasi (TIK). 

    Perancangan dan pembuatan sebuah \emph{software} atau aplikasi tidak lepas dari penerapan \emph{Software Development Life Cycle} (SDLC). SDLC bertujuan untuk memaksimalkan output sistem yang dibuat, baik itu dari segi kualitas yang tinggi sesuai dengan ekspektasi stakeholder dan para pengguna \cite{sdlc} dikarenakan SDLC berperan dalam perencanaan, kontrol, transparansi, meminimalisir risiko, dan biaya dalam sebuah pengerjaan proyek \cite{sdlc2}. Ada banyak jenis SDLC yang dapat diterapkan dengan menimbang kebutuhan dan keadaan pengembang.

    Berdasarkan permasalahan di atas maka dilakukan penelitian pengembangan dengan judul \textbf{“Implementasi metode \textit{Rapid Application Development} (RAD) dalam perancangan dan pembuatan aplikasi \textit{mobile} KebunQ BPP Lampung”}.  Penelitian ini dilakukan sekaligus untuk membantu program \textit{Low Cost Smart Farming} BPP Lampung. Pemilihan metode RAD pada penelitian ini didasarkan atas ketersediaan waktu pengerjaan yang pendek \cite{Sukamto} dan jumlah tim pengembangan yang kecil \cite{jurnal empiris}.
\\

    \section{Rumusan Masalah}
      Berdasarkan latar belakang yang telah dipaparkan, maka rumusan masalahnya adalah bagaimana implementasi metode RAD dalam perancangan dan pembuatan aplikasi \textit{mobile} KEBUNQ?
      \\
    \section{Tujuan Penelitian}
      Adapun tujuan dalam penelitian ini adalah mengimplementasikan metode RAD dalam merancang dan membuat aplikasi \textit{mobile} KEBUNQ.
      \\
    \section{Batasan Masalah}
      Adapun batasan masalah dalam penelitian ini adalah penelitian perancangan dan pembuatan aplikasi KEBUNQ ini dibatasi pada pengoperasian aplikasi di sistem operasi android.
      \\
    \section{Manfaat Penelitian}
    Penelitian ini diharapkan dapat bermanfaat bagi BPP Lampung dalam \textit{monitoring} dan \textit{controlling} pengelolaan lahan yang lebih efektif serta berguna dalam berjalannya program low cost smart farming BPP Lampung.
    \\


    \section{Sistematika Penulisan}
    Pada penelitian ini peneliti menyusun berdasarkan sistematika penulisan sebagai berikut: 
      \begin{itemize}
        \item BAB I PENDAHULUAN
        \\
        Bab ini berisi tentang latar belakang, rumusan masalah, tujuan penelitian, 
        batasan masalah, manfaat penelitian, dan sistematika penulisan.

        \item BAB II TINJAUAN PUSTAKA
        \\
          Bab ini berisi tentang teori-teori yang mendukung atau berhubungan dengan aplikasi ini.
        \item BAB III METODE PENELITIAN
        \\
          Bab ini berisi tentang teori-teori yang mendukung atau berhubungan dengan aplikasi ini.
        \item BAB IV HASIL IMPLEMENTASI DAN PEMBAHASAN
        \\
          Bab ini berisi tentang hasil implementasi dari rancangan penelitian beserta pembahasannya.
        \item BAB V KESIMPULAN DAN SARAN
        \\
          Bab ini berisi kesimpulan dan saran yang didapatkan dari penelitian ini.
        \\
      \end{itemize}

  \end{justify}

\end{flushleft}
\newpage