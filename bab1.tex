%-----------------------------------------------------------------------------%
\chapter{PENDAHULUAN}
%-----------------------------------------------------------------------------%

\vspace{4.5pt}

\begin{flushleft}
    \section{Latar Belakang} 

\begin{justify}
  \noindent 
  
  KebunQ buat apa, urgensi, data pendukung\\
  data lampung, luas, butuh efisiensi, bagaimana pertanian di lampung.\\
  rumusan masalah dapat mencaapai tujuan dan untuk mencapai tujuan menggunakan metoda yang diteliti\\
  dengan metode diharapkan ..... manfaat penelitian.
  \\
  \\
  Berdasarkan masalah yang ada maka perlu dirancang dan dibuat aplikasi untuk membantu
\textit{monitoring} data \textit{realtime} dan kontrol berdasarkan sensor dan aktuator yang tersedia pada alat serta database yang dapat menyimpan data rekaman nilai sensor untuk kebutuhan
penelitian BPP Lampung kedepannya maupun pihak yang bekerjasama dengan BPP Lampung. \textit{Software Development Life Cycle} (SDLC) yang diimplementasikan adalah
\textit{Rapid Application Development} (RAD). Penggunaan RAD dalam perancangan dan pembuatan aplikasi KebunQ ini merupakan gagasan yang tepat karena sangat cocok untuk 
waktu pengerjaan yang pendek [McLeod, Management Information Systems. 2001] dan resources team yang terbatas.

\end{justify}
\vspace{1cm}   

\section{Rumusan Masalah}
\begin{justify}
  Berdasarkan latar belakang yang telah dipaparkan, maka rumusan masalahnya adalah sebagai berikut.
\end{justify}
\begin{enumerate}
  \item Bagaimana penerapan metode RAD dalam perancangan dan pembuatan aplikasi mobile KebunQ?
  \item Bagaimana membuat aplikasi yang dapat berjalan di andoid?
  \item Bagaimana integrasi antara aplikasi KebunQ dengan alat yang ada di lahan?
  
  
\end{enumerate}
\vspace{2cm}

\section{Tujuan Penelitian}

\begin{justify}
  Adapun tujuan dalam penelitian ini adalah sebagai berikut.


\end{justify}
\begin{enumerate}
  \item Menganalisa penerapan RAD dalam pembuatan aplikasi KEBUNQ.
  \item Merancang dan membuat aplikasi KEBUNQ.
  \item Menganalisa integrasi antara aplikasi, alat dan database.
  
  \end{enumerate}
\vspace{2cm}

\section{Batasan Masalah}

\begin{justify}
  Adapun batasan masalah dalam penelitian ini adalah sebagai berikut.


\end{justify}
\begin{enumerate}
  \item Pembahasan hanya berfokus pada perancangan dan pembuatan aplikasi mobile KEBUNQ.
  \item Pengerjaan backend dan alat dikerjakan oleh individu yang berbeda dalam team ini.
  \item Bahasa pemrograman yang digunakan adalah bahasa dart dengan framework flutter.
  \item Release aplikasi yang dibuat hanya terbatas release pada playstore belum pada appstore walaupun pembuatan menggunakan menggunakan bahasa \textit{crossplatform}.
  \item Berhubung aplikasi KEBUNQ merupakan project instansi terkait, maka beberapa data tidak dicantumkan pada penelitian ini.

  \end{enumerate}
\vspace{2cm}


\section{Manfaat Penelitian}

\begin{justify}
  Adapun manfaat dari penelitian ini adalah sebagai berikut.


\end{justify}
\begin{enumerate}
  \item Membantu instansi terkait dalam mengelola lahan pertanian.
  \item Dalam pengelolaan lahan, data seperti rekaman nilai sensor diperlukan untuk monitoring kualitaslahan. Dengan adanya aplikasi KEBUNQ, instansi terkait dapat memonitori lahan secara realtime dan data rekamannya dapat digunakan pada penelitian yang dilaksanakan instansu terkait.

  \end{enumerate}
\vspace{2cm}



\section{Sistematika Penulisan}
Pada penelitian ini peneliti menyusun berdasarkan sistematika penulisan sebagai berikut: \\
\begin{itemize}
  \item \noindent BAB I PENDAHULUAN
  \begin{justify}
  Bab ini berisi tentang latar belakang, rumusan masalah, tujuan penelitian, 
  batasan masalah, manfaat penelitian, dan sistematika penulisan.
  \end{justify}
\end{itemize}

\begin{itemize}
  \item \noindent BAB II TINJAUAN PUSTAKA
  \begin{justify}
    Bab ini berisi tentang teori-teori yang mendukung atau berhubungan dengan aplikasi ini.

  \end{justify}
\end{itemize}

\begin{itemize}
  \item \noindent BAB III METODE PENELITIAN
  \begin{justify}
    Bab ini berisi tentang teori-teori yang mendukung atau berhubungan dengan aplikasi ini.

  \end{justify}
\end{itemize}


 \noindent \textbf{BAB I \hspace{1cm} PENDAHULUAN}
\begin{addmargin}[2.35cm]{0em}
Bab ini berisi tentang latar belakang, idenrumusan masaltifikasi masalah, tujuan penelitian, manfaat penelitian, batasan masalah, metode penelitian, dan sistematika penulisan.
\end{addmargin}
\noindent \textbf{BAB II \hspace{0.8cm} LANDASAN TEORI}
\begin{addmargin}[2.35cm]{0em}
Bab ini berisi tentang teori-teori yang mendukung atau berhubungan denga aplikasi ini.
\end{addmargin}
\noindent \textbf{BAB III \hspace{0.7cm} ANALISA DAN PERANCANGAN SISTEM}
\begin{addmargin}[2.35cm]{0em}
Bab ini menjelaskan tentang proses menganalisa dan merancang sistem aplikasi ini.
\end{addmargin}
\noindent \textbf{BAB IV \hspace{0.7cm} IMPLEMENTASI DAN PENGUJIAN}
\begin{addmargin}[2.35cm]{0em}
Bab ini berisi tentang implementasi dan pengujian sistem aplikasi yang telah dibuat.
\end{addmargin}
\noindent \textbf{BAB V \hspace{0.8cm} KESIMPULAN DAN SARAN}
\begin{addmargin}[2.35cm]{0em}
Bab ini berisi tentang kesimpulan dan saran untuk mendukung perbaikan sistem aplikasi ini.
\end{addmargin}


\end{flushleft}



\newpage