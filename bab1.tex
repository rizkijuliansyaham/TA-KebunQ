%-----------------------------------------------------------------------------%
\chapter{PENDAHULUAN}
%-----------------------------------------------------------------------------%

\vspace{4.5pt}

\section{Latar Belakang} \label{sec:latar_belakang}
\blindtext

\blindtext
\section{Identifikasi Masalah}
Berdasarkan latar belakang di atas dapat diidentifikasi masalah-masalah sebagai berikut:
\begin{enumerate}[nolistsep,leftmargin=0.5cm]
\item Masalah 1.
\item Masalah 2.
\item Masalah 3.
\end{enumerate}

\section{Rumusan Masalah}

\section{Tujuan Penelitian}

\section{Manfaat Penelitian}
\begin{enumerate}[nolistsep,leftmargin=0.5cm]
\item Manfaat bagi pribadi.
\item Manfaat bagi institusi.
\item Manfaat bagi masyarakan umum.
\end{enumerate}

\section{Batasan Masalah}
Dalam penelitian ini, peneliti akan membatasi masalah yang akan diteliti antara lain:
\begin{enumerate}[nolistsep,leftmargin=0.5cm]
\item Batasan 1.
\item Batasan 2.
\item Batasan 3.
\end{enumerate}


\section{Metode Penelitian}

\subsection{Pengumpulan Data}
Pengumpulan data dilakukan untuk memperoleh informasi yang
dibutuhkan dan merupakan salah satu hal yang mempengaruhi kualitas data hasil penelitian. Data penelitian sendiri diperoleh:
\begin{enumerate}[nolistsep,leftmargin=0.5cm]
\item Observasi \\
BlablablablblablalbalalBlablablablblablalbalal Blablablablblablalbalal Blablablablblablalbalal Blablablablblablalbalal Blablablablblablalbalal
\item Wawancara \\
Blablablablblablalbalal Blablablablblablalbalal Blablablablblablalbalal vBlablablablblablalbalal Blablablablblablalbalal Blablablablblablalbalal Blablablablblablalbalal
\item Studi Literatur \\
Blablablablblablalbalal Blablablablblablalbalal Blablablablblablalbalal Blablablablblablalbalal
\end{enumerate}

\subsection{Pengembangan Sistem}

\subsection{Sistematika Penulisan}
Pada penelitian ini peneliti menyusun berdasarkan sistematika penulisan sebagai berikut: \\
\noindent \textbf{BAB I \hspace{1cm} PENDAHULUAN}
\begin{addmargin}[2.35cm]{0em}
Bab ini berisi tentang latar belakang, identifikasi masalah, tujuan penelitian, manfaat penelitian, batasan masalah, metode penelitian, dan sistematika penulisan.
\end{addmargin}
\noindent \textbf{BAB II \hspace{0.8cm} LANDASAN TEORI}
\begin{addmargin}[2.35cm]{0em}
Bab ini berisi tentang teori-teori yang mendukung atau berhubungan denga aplikasi ini.
\end{addmargin}
\noindent \textbf{BAB III \hspace{0.7cm} ANALISA DAN PERANCANGAN SISTEM}
\begin{addmargin}[2.35cm]{0em}
Bab ini menjelaskan tentang proses menganalisa dan merancang sistem aplikasi ini.
\end{addmargin}
\noindent \textbf{BAB IV \hspace{0.7cm} IMPLEMENTASI DAN PENGUJIAN}
\begin{addmargin}[2.35cm]{0em}
Bab ini berisi tentang implementasi dan pengujian sistem aplikasi yang telah dibuat.
\end{addmargin}
\noindent \textbf{BAB V \hspace{0.8cm} KESIMPULAN DAN SARAN}
\begin{addmargin}[2.35cm]{0em}
Bab ini berisi tentang kesimpulan dan saran untuk mendukung perbaikan sistem aplikasi ini.
\end{addmargin}

\newpage