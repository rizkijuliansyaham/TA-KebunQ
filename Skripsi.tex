% Tipe dokumen adalah report dengan satu kolom. 
% Menggatur setting halaman 
\documentclass[12pt, a4paper, onecolumn, oneside, final, onehalfspacing]{report}

% Load konfigurasi LaTeX untuk tipe laporan thesis
\usepackage{unpam}
\usepackage{enumitem}
\usepackage{ragged2e}
\usepackage{color}
\usepackage{hyperref}
\usepackage{setspace}


% Daftar pemenggalan suku kata dan istilah dalam LaTeX
\include{hype.indonesia}

% Variabel baru untuk menyimpan nomor halaman
\newcounter{originalpagenumber}%

% Awal bagian penulisan laporan
\begin{document}

% Sampul Laporan
\begin{titlepage}
\begin{center}
	

	\onehalfspacing
	\large \bfseries IMPLEMENTASI METODE \textit{RAPID APPLICATION DEVELOPMENT} (RAD) DALAM PERANCANGAN DAN PEMBUATAN APLIKASI MOBILE KEBUNQ BPP LAMPUNG\\
	\vspace{2cm}
	 \large TUGAS AKHIR \\
	 \normalfont \normalsize Diajukan sebagai syarat menyelesaikan jenjang strata Satu (S-1) di Program Studi Teknik Informatika, Jurusan Teknologi, Produksi dan Industri, Institut Teknologi Sumatera\\
	 
	\vspace{1cm}
	
	
	\large Oleh : \\
	RIZKI JULIANSYAH \\
	14116151
	
	\vspace{1cm}
	\includegraphics[width=5cm]{images/logo-itera.png}\\
	\vspace{2cm}
	
	% \normalsize PROGRAM STUDI TEKNIK INFORMATIKA \\
	\large PROGRAM STUDI TEKNIK INFORMATIKA \\
	JURUSAN TEKNOLOGI PRODUKSI DAN INDUSTRI \\
	INSTITUT TEKNOLOGI SUMATERA \\
	LAMPUNG SELATAN \\
	2022
	

	
	
\end{center}

\end{titlepage}

\newpage

% Daftar isi, gambar, dan tabel
% Gunakan penomeran Romawi (i, ii, iii, ...) setelah bagian ini.

\pagenumbering{roman}

% Judul Laporan
\begin{center}
	\onehalfspacing
	\large \bfseries PERANCANGAN SISTEM INFORMASI KOPERASI SIMPAN PINJAM BERBASIS WEB \\
	\vspace{1cm}
	 \large SKRIPSI \\
	 \normalsize Diajukan Untuk Melengkapi Salah Satu Syarat \\ Memperoleh Gelar Sarjana Komputer
	
	\vspace{2cm}
	
	\includegraphics[width=3cm]{images/logo-itera.png}
	
	\vspace{1cm}
	\large OLEH: \\
	RIZKI JULIANSYAH \\
	14116151
	
	\vspace{3cm}
	
	\normalsize PROGRAM STUDI TEKNIK INFORMATIKA \\
	\large FAKULTAS TEKNIK \\
	UNIVERSITAS PAMULANG \\
	2014
	

	
\end{center}

\newpage

% Lembar Pengesahan
\phantomsection \addcontentsline{toc}{chapter}{LEMBAR PENGESAHAN}
%
% Lembar persetujuan
\doublespacing
\chapter*{\uppercase{LEMBAR PENGESAHAN}}

\renewcommand{\arraystretch}{1.5}
\noindent Tugas Akhir dengan judul “Implementasi Metode \textit{Rapid Application Development} (RAD) Dalam Perancangan dan Pembuatan Aplikasi Mobile KebunQ BPP Lampung” adalah benar dibuat oleh saya sendiri dan belum pernah dibuat dan diserahkan sebelumnya, baik sebagian ataupun seluruhnya, baik oleh saya ataupun orang lain, baik di Institut Teknologi Sumatera maupun di institusi pendidikan lainnya.
% \begin{table}[ht]
% \begin{adjustwidth}{-0.15cm}{}
% 	\begin{tabularx}{\textwidth}{llX}
% 		Nama & : & Imron Rosdiadna \\
% 		NIM & : & 2010140419 \\
% 		Program Studi & : & Teknik Informatika \\
% 		Fakultas & : & Teknik \\
% 		Jenjang Pendidikan & : & Strata 1 \\
% 		Judul Skripsi & : & PERANCANGAN SISTEM INFORMASI KOPERASI SIMPAN PINJAM BERBASIS WEB
% 	\end{tabularx}
% \end{adjustwidth}
% \end{table}


\noindent
\vspace{0.3cm}
\begin{tabularx}{\linewidth}{XX}
\begin{minipage}{\linewidth}
	\noindent Lampung Selatan, DD-MM-YYYY
\vspace{2cm}
\noindent Penulis,\\
Rizki Juliansyah\\
NIM. 14116151
\end{minipage} &
\begin{minipage}{\linewidth}\centering

Foto
\end{minipage}
\end{tabularx}

\vspace{1cm}
\onehalfspacing
\centering\noindent Diperiksa dan disetujui oleh,

\noindent
\vspace{0.3cm}
\begin{tabularx}{\linewidth}{XX}
\begin{minipage}{\linewidth}
	\noindent Pembimbing\\
% \vspace{2cm}
\noindent 1. Nama Pembimbing,\\
NIP. XXXXX
\end{minipage} &

\begin{minipage}{\linewidth}\centering
\noindent
Tanda Tangan
\end{minipage}
\end{tabularx}

\vspace{2.5cm}
\begin{center}
\begin{minipage}{\linewidth}\centering

\vspace{1cm}
Disahkan oleh, \\
Koordinator Program Studi Teknik Informatika \\
Jurusan Teknologi, Produksi dan Industri \\
Institut Teknologi Sumatera\\
\vspace{2.5cm}
\underline{Kaprodi, S.Si, M.Si} \\
NIP. XXXXXXX
\end{minipage}
\end{center}

\newpage
\onehalfspacing


% Lembar Pernyataan
\phantomsection \addcontentsline{toc}{chapter}{LEMBAR PERNYATAAN}
\include{pernyataan}

% Lembar Persetujuan
\phantomsection \addcontentsline{toc}{chapter}{LEMBAR PERSETUJUAN}
\include{persetujuan}

% persembahan
\phantomsection \addcontentsline{toc}{chapter}{PERSEMBAHAN}
\include{persembahan}

% Kata Pengantar
\phantomsection \addcontentsline{toc}{chapter}{KATA PENGANTAR}
% Kata Pengantar
\chapter*{KATA PENGANTAR}

% \blindtext \\[2cm]
\begin{justify}
\noindent Puji syukur kehadirat Allah SWT atas limpahan rahmat, karunia, serta petunjuk- Nya sehingga penyusunan tugas akhir ini telah terselesaikan dengan baik. Dalam penyusunan tugas akhir ini penulis telah banyak mendapatkan arahan, bantuan, serta dukungan dari berbagai pihak. Oleh karena itu pada kesempatan ini penulis mengucapan terima kasih kepada:
\end{justify}
\hfill
\begin{minipage}[t]{4.9cm}
\centering
	Pamulang, 01 Agustus 2014 \\ [2cm]
	Imron Rosdiana
\end{minipage}

\newpage

% % ringkasan
% \phantomsection \addcontentsline{toc}{chapter}{RINGKASAN}
% \include{ringkasan}

% Lembar Abstrak
\phantomsection \addcontentsline{toc}{chapter}{ABSTRAK}

\chapter*{ABSTRAK}
\noindent sa

\begin{singlespace}
    \begin{justify}
        Ini\\[2cm]
        Kata Kunci: \textit{Sistem Informasi, Testing Project}
    \end{justify}

\end{singlespace}

\newpage

% Lembar Abstract
\phantomsection \addcontentsline{toc}{chapter}{ABSTRACT}

% Halaman Abstract

\chapter*{ABSTRACT}

\begin{singlespace}
% \blindtext \\[20pt]
\begin{justify}
    Halaman ABSTRAK berisi uraian tentang latar belakang, tujuan, metodologi penelitian, hasil / kesimpulan. Ditulis dalam BAHASA INDONESIA tidak lebih dari 250 kata, dengan jarak antar baris satu spasi.
Pada akhir abstrak ditulis kata “Kata Kunci” yang dicetak tebal, diikuti tanda titik dua dan kata kunci yang tidak lebih dari 5 kata. Kata kunci terdiri dari kata-kata yang khusus menunjukkan dan berkaitan dengan bahan yang diteliti, metode/instrumen yang digunakan, topik penelitian. Kata kunci diketik pada jarak dua spasi dari baris akhir isi abstrak.
\end{justify}
Keywords: \textit{Information System, Testing Project}
\end{singlespace}

\newpage





\vspace*{-2.5cm}
\tableofcontents
\phantomsection
\addcontentsline{toc}{chapter}{DAFTAR ISI}
\clearpage
\vspace*{-2.5cm}
\listoftables
\phantomsection
\addcontentsline{toc}{chapter}{DAFTAR TABEL}
\clearpage
\vspace*{-2.5cm}
\listoffigures
\phantomsection
\addcontentsline{toc}{chapter}{DAFTAR GAMBAR}
\clearpage

% KURANG DAFTAR LAMPIRAN

% Gunakan penomeran Arab (1, 2, 3, ...) setelah bagian ini.
\pagenumbering{arabic}

%
% Atur header dan footer dalam dokumen.
% 
\renewcommand{\headrulewidth}{0.0pt}
	\fancyhf{} 
	\fancyhead[L]{} 
	\fancyhead[C]{} 
	\fancyhead[R]{\thepage} 
	\renewcommand{\headrulewidth}{0.0pt}
	\renewcommand{\footrulewidth}{0.0pt} 
\pagestyle{fancy}


%-----------------------------------------------------------------------------%

\chapter{PENDAHULUAN}
%-----------------------------------------------------------------------------%

\vspace{4.5pt}

\begin{flushleft}
  \begin{justify}
    \section{Latar Belakang} 
    Sektor pertanian merupakan sumber daya alam yang seharusnya dikelola dengan sebaik-baiknya. 
    Pengelolaan sektor pertanian yang baik dipengaruhi oleh penggunaan teknologi yang tepat guna dan keefektifan dalam pengoperasiannya. 
    Penggunaan teknologi yang tepat guna dipengaruhi oleh beberapa aspek lokal, diantaranya adalah aspek lingkungan, aspek sosial (sumber daya manusia lokal), dan aspek ekonomi masyarakat \cite{dokumenBalitbang,teknologi}. Namun, pengoperasian teknologi pada sektor pertanian beberapa diantaranya masih 
    memakan waktu yang lama dan menggunakan tenaga kerja manual. Urgensi pengoperasian teknologi yang efektif mempengaruhi produktivitas 
    pertanian, yaitu mempermudah pekerjaan petani sehingga memakan waktu yang tidak lama serta tidak dibutuhkannya lagi tenaga kerja manual. 

    Berdasarkan hasil wawancara terhadap penanggungjawab program smart farming BPP Lampung, menyatakan bahwa pengolahan lahan di 
    Balai Pelatihan Pertanian (BPP) Lampung mengharuskan tenaga kerja datang ke lokasi untuk melakukan \textit{monitoring} 
    kondisi lahan, diantaranya: pengecekan suhu udara, kelembapan udara, intensitas cahaya, suhu air, suhu tanah, ppm air, 
    pH tanah, pH air, kelembapan tanah, dan tekanan udara menggunakan alat pengukur. Selain \textit{monitoring}, dilakukan juga 
    kontrol sistem penyiraman pada lahan. Sistem \textit{monitoring} dan \textit{controlling} tersebut tergolong tidak efektif 
    dikarenakan masih beroperasi menggunakan tenaga kerja manual sehingga memakan waktu yang lama. Maka daripada itu diperlukannya  
    inovasi yang dapat mendukung keefektifan para petani dalam mengoperasikan teknologi. 

    \textit{Smartphone} merupakan teknologi yang berkembang pesat saat ini. Dalam referensi \cite{web-datasmartphone} 
    jumlah pengguna smartphone di Indonesia mencapai 170,4 juta. Jumlah petani per 2019 dalam catatan Badan Pusat Statistik (BPS) mencapai 33,4 juta orang \cite{databps}. 
    Jumlah petani di Indonesia akan terus bertambah mengingat perekonomian nasional sangat bergantung pada sektor pertanian sesuai dengan 
    referensi \cite{jurnal-kajianAplikasi} yang menyatakan bahwa 14,9\% dari Produk Domestik Bruto (PDB) berasal dari sektor pertanian. Berdasarkan data tersebut, ketersediaan \textit{smartphone} di kalangan petani Indonesia dapat memberikan dampak positif yaitu peningkatan produktivitas pertanian melalui penerapan Teknologi Informasi dan  Komunikasi (TIK). 

    Perancangan dan pembuatan sebuah \emph{software} atau aplikasi tidak lepas dari 
    penerapan \emph{Software Development Life Cycle} (SDLC). SDLC bertujuan untuk memaksimalkan 
    output \emph{software} yang dibuat, baik itu dari segi kualitas yang tinggi sesuai dengan ekspektasi 
    stakeholder dan para pengguna \cite{sdlc} dikarenakan SDLC berperan dalam perencanaan, 
    kontrol, transparansi, meminimalisir risiko, dan biaya dalam sebuah pengerjaan proyek \cite{sdlc2}. 
    Ada banyak jenis SDLC yang dapat diterapkan dengan menimbang kebutuhan dan keadaan pengembang, diantaranya \emph{waterfall} \cite{waterfall}, \emph{Agile} \cite{agile}, dan \emph{Rapid Application Development} (RAD) \cite{Sukamto}. Berdasarkan referensi \cite{Sukamto} RAD merupakan metode pengembangan yang cepat dikarenakan pengerjaan dapat dilakukan secara paralel oleh tim lain atau anggota lain dalam tim.

    Berdasarkan permasalahan di atas maka dilakukan penelitian pengembangan dengan judul \textbf{“Implementasi metode \textit{Rapid Application Development} (RAD) dalam perancangan dan pembuatan aplikasi \textit{mobile} KebunQ BPP Lampung”}.  Penelitian ini dilakukan sekaligus untuk membantu program \textit{Low Cost Smart Farming} BPP Lampung. Pemilihan metode RAD pada penelitian ini didasarkan atas ketersediaan waktu pengerjaan yang pendek \cite{Sukamto} dan jumlah tim pengembangan yang kecil \cite{jurnal empiris}.
\\

    \section{Rumusan Masalah}
      Berdasarkan latar belakang yang telah dipaparkan, maka rumusan masalahnya adalah sebagai berikut:
      \begin{enumerate}
        \item Bagaimana implementasi metode RAD dalam perancangan dan pembuatan aplikasi \textit{mobile} KEBUNQ?
        \item Bagaimana pengujian yang dilakukan pada aplikasi KEBUNQ?\\
      \end{enumerate}
    \section{Tujuan Penelitian}
      Adapun tujuan dalam penelitian ini adalah mengimplementasikan metode RAD dalam merancang dan membuat aplikasi \textit{mobile} KEBUNQ.
      \\
    \section{Batasan Masalah}
      Adapun batasan masalah dalam penelitian ini adalah sebagai berikut:
      \begin{enumerate}
        \item Penelitian perancangan dan pembuatan aplikasi KEBUNQ ini dibatasi pada pengoperasian aplikasi di sistem operasi android.
        \item Lahan yang diamati dan diimplementasikan aplikasi KEBUNQ dalam penelitian ini terbatas pada lahan cabai besar, lahan cabai kecil, dan \emph{screen} hidroponik di Balai Pelatihan Pertanian Lampung.\\
      \end{enumerate} 
    \section{Manfaat Penelitian}
    Penelitian ini diharapkan dapat bermanfaat bagi BPP Lampung dalam \textit{monitoring} dan \textit{controlling} pengelolaan lahan yang lebih efektif serta berguna dalam berjalannya program low cost smart farming BPP Lampung.
    \\


    \section{Sistematika Penulisan}
    Pada penelitian ini peneliti menyusun berdasarkan sistematika penulisan sebagai berikut: 
      \begin{itemize}
        \item BAB I PENDAHULUAN
        \\
        Bab ini berisi tentang latar belakang, rumusan masalah, tujuan penelitian, 
        batasan masalah, manfaat penelitian, dan sistematika penulisan.

        \item BAB II TINJAUAN PUSTAKA
        \\
          Bab ini berisi tentang teori-teori yang mendukung atau berhubungan dengan aplikasi ini.
        \item BAB III METODE PENELITIAN
        \\
          Bab ini berisi tentang teori-teori yang mendukung atau berhubungan dengan aplikasi ini.
        \item BAB IV HASIL IMPLEMENTASI DAN PEMBAHASAN
        \\
          Bab ini berisi tentang hasil implementasi dari rancangan penelitian beserta pembahasannya.
        \item BAB V KESIMPULAN DAN SARAN
        \\
          Bab ini berisi kesimpulan dan saran yang didapatkan dari penelitian ini.
        \\
      \end{itemize}

  \end{justify}

\end{flushleft}
\newpage
%-----------------------------------------------------------------------------%
\chapter{LANDASAN TEORI}
%-----------------------------------------------------------------------------%

%
\vspace{4.5pt}

\begin{flushleft}
    \section{Tinjauan Pustaka}
    \section{Dasar Teori}
    \begin{justify}
        \subsection{Sistem \textit{Monitoring}}
\blindtext

\subsection{Sistem Kontrol}


        \subsection{Aplikasi Mobile}

        \subsection{\textit{Rapid Application Development} (RAD)}

        \subsection{Flutter}

        \subsection{API}

        % \subsection{VPS}

        \subsection{Database}

        \subsection{\textit{Use Case} Diagram}

        \subsection{\textit{Black Box Testing}}

        \subsection{\textit{User Acceptance Testing} (UAT)}
        \noindent User Acceptance Testing (UAT) adalah pengujian terakhir yang dilakukan oleh user secara langsung, dan pada saat pengujian berlangsung pembuatan dokumen juga dilakukan sebagai bukti penerimaan sistem oleh pengguna. 
        \textcolor{red}{[Mutiara, A. B., Awaludin, R., Muslim, A. and T. Oswari, “Testing Implementasi Website Rekam Medis Elektronik Opeltgunasys Dengan Metode Acceptance Testing,” 2014.].
        }
        

        \subsection{Skala \textit{Likert}}

    \end{justify}



\end{flushleft}



\newpage
%-----------------------------------------------------------------------------%
\chapter{METODE PENELITIAN}
%-----------------------------------------------------------------------------%

%
\vspace{4.5pt}

\begin{flushleft}
   \begin{justify}
      \section{Alur Penelitian}
      Langkah-langkah penelitian yang akan dilakukan sebagai bagian dari penelitian ini dapat dilihat pada Gambar 3.1
      \begin{figure}[ht]
         \centering
         \includegraphics[width=6cm]{images/alur_penelitian.png}\\
         \caption{Alur Penelitian}
     \end{figure}

      \section{Penjabaran Langkah Penelitian}
       Berikut ini merupakan prosedur penelitian yang dilakukan.
         \subsection{Studi Literatur}
         Mencari dan mengumpulkan referensi yang berkaitan dengan penelitian melalui media buku, jurnal dan e-book.\\
      
         \subsection{Observasi}
         Melakukan pengamatan di Balai Pelatihan Pertanian (BPP) Lampung terkait sistem pengolahan lahan cabai dan greenhouse.\\
         \subsection{RAD}
         Melakukan perancangan dan pembuatan aplikasi KEBUNQ dengan mengikuti langkah proses yang tercantum dalam \textit{Rapid Application Development} (RAD). Pada langkah ini dilakukan lima tahapan yaitu, (1) Pemodelan Bisnis, (2) Pemodelan Data, (3) Pemodelan Proses, (4) Pembuatan Aplikasi, dan (5) Pengujian.\\
         \subsection{Uji Lapangan}
         Melakukan pengujian aplikasi KEBUNQ dengan alat yang terpasang pada lahan.\\
         \subsection{Kesimpulan}
         Melakukan analisa dan menulis kesimpulan dari penelitian ini.\\

       \section{Alat dan Bahan Tugas Akhir}
         \subsection{Alat}
         Alat yang digunakan dalam penelitian ini.
         \begin{enumerate}
            \item Macbook Pro (13-\textit{inch}, 2016, \textit{Four Thunderbolt 3 Ports}) dengan OS Monterey \textit{Version} 12.3.1 (21E258), \textit{processor} 2,9 GHz Dual-Core Intel Core i5, \textit{memory} 8 GB 2133 MHz LPDDR3, \textit{graphics} Intel Iris Graphics 550 1536 MB
            \item \textit{Smartphone} dengan spesifikasi minimum OS Android 6.0 (\textit{marshmallow}). Pada penelitian ini digunakan untuk melakukan \textit{testing} dalam proses pembuatan aplikasi
            \item Visual Studi Code digunakan sebagai \textit{code editor} dalam pemrograman
            \item Postman digunakan sebagai alat bantu dalam melakukan \textit{testing} API
            \item Figma dan Inkscape digunakan sebagai alat dalam pembuatan \textit{User Interface Layout} dan \textit{assets}\\
         \end{enumerate}
         \subsection{Bahan}
         \begin{enumerate}
            \item Dokumen \textit{Software Requirements Specification} sebagai standar dan batasan dalam pengembangan aplikasi KEBUNQ
            \item Data Kuesioner yang diisi saat pengujian aplikasi\\
         \end{enumerate}
      \section{Metode Tugas Akhir}
      Metode yang digunakan dalam pengerjaan tugas akhir ini
      \begin{enumerate}
         \item Alur pengembangan tugas akhir.
         \begin{figure}[ht]
            \centering
            \includegraphics[width=7cm]{images/UI/Frame 1.png}
            \caption{\textit{Flow Chart} Alur Pengembangan Tugas Akhir}
        \end{figure}
         \item Metode pengembangan yang digunakan adalah \textit{Rapid Application Development} (RAD)
         \item Cara pengumpulan data yang digunakan adalah kuesioner dan pengujian\\
      \end{enumerate}

      \section{Ilustrasi Perhitungan Metode}

      \section{Rancangan Pengujian}

   
   \end{justify}
   
\end{flushleft}

% \vspace{5cm}
% \noindent \textbf{CONTOH Penulisan}
% \section{Analisa Sistem}

% \subsection{Analisa Sistem Saat Ini}
% Analisa sistem pendukung keputusan dalam penentuan penjurusan dibuat oleh peneliti dalam bentuk use case diagram yang mewakili secara sederhana dan bisa dijadikan sebagai bahan dalam evaluasi sistem yang berjalan, sehingga sistem dapat terlihat tanpa harus mengetahui secara detail prosedur yang berjalan.
% \begin{figure}[ht]
% 	\centering
% 	\includegraphics[width=10cm]{images/UseCaseDiagramSistemSaatIni}
% 	\caption{Use Case Diagram Analisa Sistem Saat Ini}
% \end{figure}

% \newpage
% \noindent Dibawah ini merupakan deskripsi dari use case yang sedang berjalan:
% \begin{enumerate}[nolistsep,leftmargin=0.5cm]
% \item \textit{Admin} melihat daftar siswa.
% \item \textit{Admin} melihat nilai setiap siswa.
% \item \textit{Admin} melihat minat setiap siswa.
% \item \textit{Admin} mencetak hasil keputusan.
% \item Siswa melihat laporan penjurusan yang telah dicetak oleh \textit{admin}
% \end{enumerate}

% \subsection{Evaluasi Sistem Saat Ini}

% \begin{table}[ht]
% \centering
% \caption{Permasalahan dan Solusinya}
% \begin{tabular}{|>{\raggedright}p{5cm}|p{2.5cm}|>{\raggedright}p{5cm}|}
%  \hline
%  \multicolumn{1}{|c}{\bfseries Masalah} & \multicolumn{1}{|c|}{\bfseries Aktor} & \multicolumn{1}{c|}{\bfseries Solusi} \\ 
%   \hline
% \begin{enumerate}
%    	\item Masalah masalah masalah Masalah masalah masalah Masalah masalah masalah Masalah masalah masalah.
%    	\item Masalah masalah masalah Masalah masalah masalah Masalah masalah masalah Masalah masalah masalah.
%    	\item Masalah masalah masalah Masalah masalah masalah Masalah masalah masalah Masalah masalah masalah.
%    \end{enumerate} &
%    \begin{enumerate}
%   	\item Aktor 1
%   	\item Aktor 2
%   \end{enumerate} &
%   \begin{enumerate}
%   \item Solusi solusi solusi Solusi solusi solusi Solusi solusi solusi Solusi solusi solusi Solusi solusi solusi.
%   \item Solusi solusi solusi Solusi solusi solusi Solusi solusi solusi Solusi solusi solusi Solusi solusi solusi.
%   \item Solusi solusi solusi Solusi solusi solusi Solusi solusi solusi Solusi solusi solusi Solusi solusi solusi.
%   \end{enumerate}
%      \tabularnewline
%   \hline
%  \end{tabular}
% \end{table}

% \subsection{Model yang Diusulkan}

% \subsection{Acitivity Diagram yang Diusulkan}

% \subsection{Perancangan Prosedur Sistem}

% \subsubsection{Use Case Diagram}

% \subsubsection{Activity Diagram}
% \begin{enumerate}[nolistsep,leftmargin=0.5cm]
% \item \textit{Activity diagram} satu

% \begin{enumerate}[label=\alph*.]
% 	\item Item 1.
% 	\item Item 2.
% 	\end{enumerate}
% \item Dua
% \end{enumerate}

% \subsubsection{Class Diagram}

% \subsubsection{Sequence Diagram}

% \subsection{Perancangan Antarmuka (Interface)}

\newpage
%-----------------------------------------------------------------------------%
\chapter{HASIL PENELITIAN DAN PEMBAHASAN}
%-----------------------------------------------------------------------------%

%
\vspace{4.5pt}
\begin{flushleft}
    \section{Hasil Penelitian}
    \subsection{Data Hasil Observasi}
\vspace{5cm}
\begin{figure}[ht]
	\centering
	\includegraphics[width=10cm]{images/UseCaseDiagramSistemSaatIni}
	\caption{Gambar Observasi}
\end{figure}
Peneliti mengamati kebutuhan sensor dan kontrol yang diperlukan petani untuk diaplikasikan pada lahannya. 
% \caption{Gambar Observasi}
    \section{Analisis Hasil Penelitian}

\end{flushleft}

\vspace{5cm}
\section{Implementasi}

\subsection{Lingkungan Perangkat Lunak}

\subsection{Spesifikasi Perangkat Keras}

\subsection{Impelementasi Antarmuka}

\subsubsection{Impelementasi Halaman Utama}

\subsubsection{Implementasi Menu File}

\subsubsection{Implementasi Menu}

\subsection{Pengguna Program}

\section{Pengujian}

\subsection{Pengujian Blackbox}

\subsection{Pengujian Whitebox}

\newpage
%-----------------------------------------------------------------------------%
\chapter{KESIMPULAN DAN SARAN}
%-----------------------------------------------------------------------------%

%
\vspace{4.5pt}

\begin{flushleft}
    \section{Kesimpulan}

    \section{Saran}
        
\end{flushleft}

\newpage

% Merubah Nama Bibliografi ke Daftar Pustaka
\renewcommand{\bibname}{DAFTAR PUSTAKA}
\phantomsection
\addcontentsline{toc}{chapter}{DAFTAR PUSTAKA}
% Daftar Pustaka

\begin{thebibliography}{7}
%\bibliographystyle{unsrt}
\providecommand{\natexlab}[1]{#1}
\providecommand{\url}[1]{\texttt{#1}}
\expandafter\ifx\csname urlstyle\endcsname\relax
  \providecommand{\doi}[1]{doi: #1}\else
  \providecommand{\doi}{doi: \begingroup \urlstyle{rm}\Url}\fi


\bibitem[1]{dokumenBalitbang}
BPTP Balitbangtan Jambi, "Alat dan Mesin Pertanian Tepat Guna untuk Tanaman Padidalam Mendukung Program Peningkatan Produksi Beras Nasional (P2BN)"
\newblock 2020, Available: http://jambi.litbang.pertanian.go.id/ind/images/PDF/Kiki1.pdf

\bibitem[2]{teknologi}
Admin dispmd, “Pengertian Teknologi Tepat Guna,” \emph{Dinas Pemberdayaan Masyarakat dan Desa}, 16-May-2018. [Online]. Available: https://dispmd.bulelengkab.go.id/informasi/detail/bank\_data/pengertian-teknologi-tepat-guna-13. [Accessed: 02-Feb-2022]. 

\bibitem[3]{web-datasmartphone}
Y. Pusparisa, “Daftar Negara Pengguna smartphone Terbanyak, Indonesia urutan berapa?: Databoks,” \emph{Databoks Pusat Data Ekonomi dan Bisnis Indonesia}, 01-Jul-2021. [Online]. Available: https://databoks.katadata.co.id/datapublish/2021/07/01/daftar-negara-pengguna-smartphone-terbanyak-indonesia-urutan-berapa. [Accessed: 06-Feb-2022]. 

\bibitem[4]{databps}
A. ID, “Jumlah Petani di indonesia - grafik alinea ID,” \emph{https://data.alinea.id/}, 11-Oct-2021. [Online]. Available: https://data.alinea.id/jumlah-petani-di-indonesia-b2cCd9Bp9c. [Accessed: 08-Feb-2022]. 

\bibitem[5]{jurnal-kajianAplikasi}
Direktorat Pangan dan Pertanian, “Studi Pendahuluan Rencana Pembangunan Jangka Menengah Nasional (RPJMN) Bidang Pangan dan Pertanian 2015 – 2016”, Direktorat Pangan dan Pertanian Kementerian Perancanaan Pembangunan Nasional/Badan Perencanaan Pembangunan Nasional, 2013.
% https://journal.uii.ac.id/Snati/article/download/6236/5598

\bibitem[6]{sdlc}
C. Jessica, “Software development life cycle (SDLC): Arti, Cara Kerja, Penerapan, Dan Manfaatnya,” \emph{Glints Blog}, 17-Dec-2021. [Online]. Available: https://glints.com/id/lowongan/sdlc-software-development-life-cycle/. [Accessed: 27-May-2022]. 

\bibitem[7]{sdlc2}
F. NKD, “Pengertian, model, Dan Tahapan SDLC\&nbsp; (software development life cycle),” \emph{Web developer LOGIQUE's Blog}, 28-Apr-2021. [Online]. Available: https://www.logique.co.id/blog/2021/04/28/tahapan-sdlc/. [Accessed: 27-May-2022]. 

\bibitem[8]{waterfall}
G. W.  Sasmito, "Penerapan Metode Waterfall Pada Desain Sistem Informasi Geografis Industri Kabupaten Tegal," \emph{Jurnal Informatika:Jurnal Pengembangan IT (JPIT)}, vol. 2, no.1, Jan. 2017.

\bibitem[9]{agile}
I. T.  Kusnadi and A.  Supiandi, "Implementasi Agile Mmethode pada Sistem Informasi Penjualan Alat Olahraga Berbasis Web," \emph{Jurnal Informatika(JURIN)}, vol. 3, no.2, pp. 28-36, Maret 2021. 

\bibitem[10]{Sukamto}
Sukamto, R. A. dan Shalahudin, M.. Rekayasa Perangkat Lunak Terstruktur dan Berorientasi Objek. Bandung : Informatika Bandung, 2016.

\bibitem[11]{jurnal empiris}
P. Beynon-Davies, C. Carne, H. Mackay, and D. Tudhope, “Rapid application development (RAD): An empirical review,” \emph{European Journal of Information Systems}, vol. 8, no. 3, pp. 212, 1999. 
% https://www.researchgate.net/publication/31978101_Rapid_application_development_RAD_An_empirical_review

\bibitem[]{web waterfall}
M. P. Putri and H. Effendi, “Implementasi metode RAD Pada website  service guide ‘Tour waterfall south sumatera’,” \emph{Jurnal Sisfokom (Sistem Informasi dan Komputer)}, vol. 7, no. 2, pp. 130–136, Sep. 2018. 
% https://www.google.com/url?sa=t&rct=j&q=&esrc=s&source=web&cd=&cad=rja&uact=8&ved=2ahUKEwjioNOUoeP3AhW88HMBHS9qCqkQFnoECAYQAQ&url=http%3A%2F%2Fjurnal.atmaluhur.ac.id%2Findex.php%2Fsisfokom%2Farticle%2Fview%2F00021&usg=AOvVaw2jJicE3WeLE5gyXS2aFEdB

\bibitem[]{jurnal RAD UAT}
A. Rahman, "Rapid Application Development Sistem Pembelajaran Daring Berbasis Android," \emph{Jurnal Intech}, vol. 1, no. 2, pp. 20-25, Nov. 2020.
% https://www.google.com/url?sa=t&rct=j&q=&esrc=s&source=web&cd=&cad=rja&uact=8&ved=2ahUKEwitj6T-oOP3AhVS7XMBHf5rDaYQFnoECAYQAQ&url=https%3A%2F%2Fjournal.unbara.ac.id%2Findex.php%2FINTECH%2Farticle%2Fdownload%2F639%2F464%2F&usg=AOvVaw3MNPdSK6Bm-OYVrWqfrTL7

\bibitem[]{jurnal RAD UAT 2}
D. Aryani, Malabay, H. D. Ariessanti, "Penerapan Rapid Application Development (RAD) Pada Perancangan Aplikasi Tracer Study Berbasis Android," \emph{eDikInformatika}, vol. 7, no. 1, pp. 111-122, Oct. 2020.
% https://digilib.esaunggul.ac.id/public/UEU-Journal-20259-11_1399.pdf

\bibitem[13]{Monitoring}
Hikmat, Dr. Harry. 2010. \emph{Monitoring dan Evaluasi Proyek}.

\bibitem[14]{Kontrol}
Katsuhiko Ogata. "Teknik kontrol automatik“, PT penerbit erlangga-SIMON. \&SCHUTER (ASUA) Pte.ltd., 1997.

\bibitem[15]{mobile}
N. Serrano, J. Hernantes and G. Gallardo. Mobile Web Apps. \emph{IEEE Software}, vol. 30, no. 5, 2013, pp. 22 -27.
% https://www.researchgate.net/publication/313868513_Studying_Mobile_Apps_for_Agriculture

\bibitem[16]{flutter}
Ganda, Yusmi P. W., Happy Flutter. Cetakan 1. Tangerang Selatan : Al Qolam, 2019.

\bibitem[17]{API}
IBM Cloud Education, “What is an application programming interface (API),” \emph{IBM}, 19-Aug-2020. [Online]. Available: https://www.ibm.com/cloud/learn/api. [Accessed: 11-Feb-2022]. 

\bibitem[18]{Database}
OCI, “What is a database?,” \emph{Oracle}, 2022. [Online]. Available: https://www.oracle.com/database/what-is-database/. [Accessed: 11-Feb-2022]. 

\bibitem[19]{Flowchart}
“What is a flowchart?,” \emph{ASQ}. [Online]. Available: https://asq.org/quality-resources/flowchart. [Accessed: 15-Feb-2022]. 

\bibitem[20]{gambar fc}
D. Rizky, “Jenis Flowchart Dan Simbol-Simbolnya,” \emph{Medium}, 30-Apr-2019. [Online]. Available: https://medium.com/dot-intern/jenis-flowchart-dan-simbol-simbolnya-ef6553c53d73. [Accessed: 20-Feb-2022]. 

\bibitem[21]{use case 1}
M. R. Adani, “Use case diagram: Pengertian, Fungsi, Teknik, Dan Contoh,” \emph{Sekawan Media | Software House \&amp; System Integrator Indonesia}, 21-Jun-2021. [Online]. Available: https://www.sekawanmedia.co.id/blog/use-case-diagram/. [Accessed: 20-Feb-2022]. 

\bibitem[22]{use case 2}
Lucid Software, “UML use case diagram tutorial,” \emph{Lucidchart}. [Online]. Available: https://www.lucidchart.com/pages/uml-use-case-diagram. [Accessed: 24-Feb-2022]. 

\bibitem[23]{figma uc}
F. Mandrelli, “UML use case diagram,” \emph{Figma}, 2021. [Online]. Available: https://www.figma.com/community/file/986330591099819762. [Accessed: 24-Feb-2022]. 

\bibitem[24]{buku scholar}
R.   Habibi   and   R.   Aprilian,   Tutorial   dan Penjelasan  Aplikasi  E-Office  Berbasis  Web Menggunakan Metode RAD, Bandung: Kreatif Industri Nusantara, 2020. 
% bukunya https://books.google.co.id/books?hl=id&lr=&id=h5PuDwAAQBAJ&oi=fnd&pg=PR1&dq=related:MMIDeyjXtEUJ:scholar.google.com/&ots=Ht-Xb71R5S&sig=X1LON_cyff00KLfgqs0woyCpU-Q&redir_esc=y#v=onepage&q&f=false
% akun beliau https://scholar.google.co.id/citations?view_op=view_citation&hl=id&user=IBtTXDAAAAAJ&citation_for_view=IBtTXDAAAAAJ:LkGwnXOMwfcC
%  halaman 101

\bibitem[25]{black box}
Iskandaria. 2012. Contoh Pengujian Black Box

\bibitem[26]{uat}
Mutiara, A. B., Awaludin, R., Muslim, A. and T. Oswari, “Testing Implementasi Website Rekam Medis Elektronik Opeltgunasys Dengan Metode Acceptance Testing,” 2014.

\bibitem[27]{likert}
Sugiyono, \emph{Memahami Penelitian Kualitatif}. Bandung, Indonesia: Alfabeta, 2012.

\bibitem[28]{kuantitatif}
Riduwan, \emph{Belajar mudah penelitian untuk guru-karyawan dan peneliti pemula}. Bandung, Indonesia: Alfabeta, 2009. 


\end{thebibliography}



%
% Lampiran 
%

\setcounter{originalpagenumber}{\number\value{page}}%
\setcounter{page}{0}
\pagenumbering{arabic}

\onehalfspacing
\begin{appendix}
	\include{markLampiran}
\end{appendix}

\pagenumbering{arabic}% 
\setcounter{page}{\number\value{originalpagenumber}}
\clearpage
\phantomsection \addcontentsline{toc}{chapter}{LAMPIRAN}
\begin{flushleft}
    \begin{justify}

    \end{justify}
\end{flushleft}



\end{document}