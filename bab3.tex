%-----------------------------------------------------------------------------%
\chapter{METODE PENELITIAN}
%-----------------------------------------------------------------------------%

%
\vspace{4.5pt}

\begin{flushleft}
   \begin{justify}
      \section{Alur Penelitian}
      Langkah-langkah penelitian yang akan dilakukan sebagai bagian dari penelitian ini dapat dilihat pada Gambar 3.1
      \begin{figure}[ht]
         \centering
         \includegraphics[width=6cm]{images/alur_penelitian.png}\\
         \caption{Alur Penelitian}
     \end{figure}

      \section{Penjabaran Langkah Penelitian}
       Berikut ini merupakan prosedur penelitian yang dilakukan.
         \subsection{Studi Literatur}
         Mencari dan mengumpulkan referensi yang berkaitan dengan penelitian melalui media buku, jurnal dan e-book.\\
      
         \subsection{Observasi}
         Melakukan pengamatan di Balai Pelatihan Pertanian (BPP) Lampung terkait sistem pengolahan lahan cabai dan greenhouse.\\
         \subsection{RAD}
         Melakukan perancangan dan pembuatan aplikasi KEBUNQ dengan mengikuti langkah proses yang tercantum dalam \textit{Rapid Application Development} (RAD). Pada langkah ini dilakukan lima tahapan yaitu, (1) Pemodelan Bisnis, (2) Pemodelan Data, (3) Pemodelan Proses, (4) Pembuatan Aplikasi, dan (5) Pengujian.\\
         \subsection{Uji Lapangan}
         Melakukan pengujian aplikasi KEBUNQ dengan alat yang terpasang pada lahan.\\
         \subsection{Kesimpulan}
         Melakukan analisa dan menulis kesimpulan dari penelitian ini.\\

       \section{Alat dan Bahan Tugas Akhir}
         \subsection{Alat}
         Alat yang digunakan dalam penelitian ini.
         \begin{enumerate}
            \item Macbook Pro (13-\textit{inch}, 2016, \textit{Four Thunderbolt 3 Ports}) dengan OS Monterey \textit{Version} 12.3.1 (21E258), \textit{processor} 2,9 GHz Dual-Core Intel Core i5, \textit{memory} 8 GB 2133 MHz LPDDR3, \textit{graphics} Intel Iris Graphics 550 1536 MB
            \item \textit{Smartphone} dengan spesifikasi minimum OS Android 6.0 (\textit{marshmallow}). Pada penelitian ini digunakan untuk melakukan \textit{testing} dalam proses pembuatan aplikasi
            \item Visual Studi Code digunakan sebagai \textit{code editor} dalam pemrograman
            \item Postman digunakan sebagai alat bantu dalam melakukan \textit{testing} API
            \item Figma dan Inkscape digunakan sebagai alat dalam pembuatan \textit{User Interface Layout} dan \textit{assets}\\
         \end{enumerate}
         \subsection{Bahan}
         \begin{enumerate}
            \item Dokumen \textit{Software Requirements Specification} sebagai standar dan batasan dalam pengembangan aplikasi KEBUNQ
            \item Data Kuesioner yang diisi saat pengujian aplikasi\\
         \end{enumerate}
      \section{Metode Tugas Akhir}
      Metode yang digunakan dalam pengerjaan tugas akhir ini
      \begin{enumerate}
         \item Alur pengembangan tugas akhir.
         \begin{figure}[ht]
            \centering
            \includegraphics[width=7cm]{images/UI/Frame 1.png}
            \caption{\textit{Flow Chart} Alur Pengembangan Tugas Akhir}
        \end{figure}
         \item Metode pengembangan yang digunakan adalah \textit{Rapid Application Development} (RAD)
         \item Cara pengumpulan data yang digunakan adalah kuesioner dan pengujian\\
      \end{enumerate}

      \section{Ilustrasi Perhitungan Metode}

      \section{Rancangan Pengujian}

   
   \end{justify}
   
\end{flushleft}

% \vspace{5cm}
% \noindent \textbf{CONTOH Penulisan}
% \section{Analisa Sistem}

% \subsection{Analisa Sistem Saat Ini}
% Analisa sistem pendukung keputusan dalam penentuan penjurusan dibuat oleh peneliti dalam bentuk use case diagram yang mewakili secara sederhana dan bisa dijadikan sebagai bahan dalam evaluasi sistem yang berjalan, sehingga sistem dapat terlihat tanpa harus mengetahui secara detail prosedur yang berjalan.
% \begin{figure}[ht]
% 	\centering
% 	\includegraphics[width=10cm]{images/UseCaseDiagramSistemSaatIni}
% 	\caption{Use Case Diagram Analisa Sistem Saat Ini}
% \end{figure}

% \newpage
% \noindent Dibawah ini merupakan deskripsi dari use case yang sedang berjalan:
% \begin{enumerate}[nolistsep,leftmargin=0.5cm]
% \item \textit{Admin} melihat daftar siswa.
% \item \textit{Admin} melihat nilai setiap siswa.
% \item \textit{Admin} melihat minat setiap siswa.
% \item \textit{Admin} mencetak hasil keputusan.
% \item Siswa melihat laporan penjurusan yang telah dicetak oleh \textit{admin}
% \end{enumerate}

% \subsection{Evaluasi Sistem Saat Ini}

% \begin{table}[ht]
% \centering
% \caption{Permasalahan dan Solusinya}
% \begin{tabular}{|>{\raggedright}p{5cm}|p{2.5cm}|>{\raggedright}p{5cm}|}
%  \hline
%  \multicolumn{1}{|c}{\bfseries Masalah} & \multicolumn{1}{|c|}{\bfseries Aktor} & \multicolumn{1}{c|}{\bfseries Solusi} \\ 
%   \hline
% \begin{enumerate}
%    	\item Masalah masalah masalah Masalah masalah masalah Masalah masalah masalah Masalah masalah masalah.
%    	\item Masalah masalah masalah Masalah masalah masalah Masalah masalah masalah Masalah masalah masalah.
%    	\item Masalah masalah masalah Masalah masalah masalah Masalah masalah masalah Masalah masalah masalah.
%    \end{enumerate} &
%    \begin{enumerate}
%   	\item Aktor 1
%   	\item Aktor 2
%   \end{enumerate} &
%   \begin{enumerate}
%   \item Solusi solusi solusi Solusi solusi solusi Solusi solusi solusi Solusi solusi solusi Solusi solusi solusi.
%   \item Solusi solusi solusi Solusi solusi solusi Solusi solusi solusi Solusi solusi solusi Solusi solusi solusi.
%   \item Solusi solusi solusi Solusi solusi solusi Solusi solusi solusi Solusi solusi solusi Solusi solusi solusi.
%   \end{enumerate}
%      \tabularnewline
%   \hline
%  \end{tabular}
% \end{table}

% \subsection{Model yang Diusulkan}

% \subsection{Acitivity Diagram yang Diusulkan}

% \subsection{Perancangan Prosedur Sistem}

% \subsubsection{Use Case Diagram}

% \subsubsection{Activity Diagram}
% \begin{enumerate}[nolistsep,leftmargin=0.5cm]
% \item \textit{Activity diagram} satu

% \begin{enumerate}[label=\alph*.]
% 	\item Item 1.
% 	\item Item 2.
% 	\end{enumerate}
% \item Dua
% \end{enumerate}

% \subsubsection{Class Diagram}

% \subsubsection{Sequence Diagram}

% \subsection{Perancangan Antarmuka (Interface)}

\newpage